\chapter{Introduction}

Lorem ipsum dolor sit amet, consectetur adipiscing elit, sed do eiusmod tempor incididunt ut labore et dolore magna aliqua. Ut enim ad minim veniam, quis nostrud exercitation ullamco laboris nisi ut aliquip ex ea commodo consequat. Duis aute irure dolor in reprehenderit in voluptate velit esse cillum dolore eu fugiat nulla pariatur. Excepteur sint occaecat cupidatat non proident, sunt in culpa qui officia deserunt mollit anim id est laborum.
I used the following stuff: \cite{A1}, \cite{DETERM_WKA}, \cite{A3}, \cite{A4}, \cite{A5}, \cite{A6}, \cite{A7}, \cite{WK_CYK}, \cite{WK_PUSHDOWN_AUT}, \cite{WK_FIN_AUT}


\chapter{Watson-Crick models and DNA}
The study of Watson-Crick models is motivated by DNA (deoxyribonucleic acid) computing. In order to study the DNA mathematically -- i.e. to perform mathematical operations, it is necessary to work with a suitable abstraction -- a model which captures its key characteristics. Specifically, there are two characterictics that the Watson-Crick models capture -- the fact that DNA is a double stranded chain and the Watson-Crick relation between DNA nucleotides.
The two fundamental models that are used to define a language in computer theory are grammars and automata. Several versions of both have been proposed but all of them work with these two characterictics in a very similar manner.

\begin{figure}
  \includegraphics[width=8cm]{placeholder.pdf}
  \centering
  \caption{The DNA double helix}
\end{figure}

DNA consists of two chains of nucleotides that are connected by covalent bonds and together form a double helix. These two chains are represented in the Watson-Crick automata by two reading heads which read two inputs independetly but controlled by the same states. Similarly, Watson-Crick grammars produce by their rules not just a chain of symbols, but two chains.

Each nucleotide contains one of the four nucleobases - cytosine (C), guanine (G), adenine (A), thymine (T). These bases are always connected with their counterpart: cytosine with guanine and adenine with thymine. That means that whenever one of the four appears in a chain, its counterpart appears in the other chain in the corresponding place being bound together by the covalent bond. The Watson-Crick models therefore introduce a complementary relation -- a relation between symbols which must be kept in the whole input for it to be valid. Typically, this relation is symmetric ($a R b \Leftrightarrow b R a$) and covers the whole alphabet (every symbol must have at least one counterpart). Often every symbol has exactly one counterpart -- just like in case of DNA (the relation is frequently defined as an identity which is still somewhat similar to the DNA pairing).

\chapter{Watson-Crick models}
Number of models working with double stranded sequences has been proposed. The purpuse of this chapter is to summarize these models and some of their key attributes that will be used in a later chapters.


\section{Watson-Crick automata}
Watson-Crick automata have been first proposed in \cite{WK_FIN_AUT} as an enhancement of standard Finite Automata. Watson-Crick finite automaton is a 6-tuple $M = (V, \rho, Q, q_0, F, P)$ with the following meaning.
\begin{itemize}
  \item{$V$ -- finite input alphabet}
  \item{$\rho \subseteq V \times V$ -- complementarity relation}
  \item{$Q$ -- finite set of states}
  \item{$q_0 \in Q$ -- starting symbol}
  \item{$F \subseteq Q$ -- set of finite states}
  \item{$P$ -- finite set of transition rules in a form $q({w_1 \atop w_2}) \rightarrow q'$ where $q, q' \in Q, w_1, w_2 \in V^*$}
\end{itemize}

Compared to Finite automata, Watson-Crick automata have different form of transition rules which read two strings at the same time. These represent the two independent reading heads -- one reading the upper strand ($w_1$) and the other reading the lower strand ($w_2$). They also add the complementarity relation which is usually required to be symmetric. The symbols in the upper and lower strands with the same indices need to adhere to it.

A Watson-Crick domain is a set $WK_{\rho}(V)$ which denotes all valid double strands associated with a given $V$ and $\rho$. Formally:
\begin{align}
	WK_{\rho}(V) = \wkdomain{V}{V}_{\rho}^{*} && \textnormal{where} && \wkdomain{V}{V}_{\rho} = \Big\{\wkdomain{a}{b} | a, b \in V, (a, b) \in \rho \Big\}
\end{align}
This implies that both strands have the same length.

A configuraion of a Watson-Crick automaton is a pair $(q, ({w_1 \atop w_2}))$ where $q \in Q$ is a current state and $w_1, w_2 \in V^*$ are the parts of the upper and lower strands yet to be read.

If $q\big({u_1 \atop u_2}\big) \rightarrow q' \in P$ and $\big({u_1 v_1 \atop u_2 v_2}\big) \in \big({V^* \atop V^*}\big)$ then $q\big({u_1 v_1 \atop u_2 v_2}\big) \Rightarrow q'\big({v_1 \atop v_2}\big)$ is a transition of the Watson-Crick automaton. $\Rightarrow^*$ denotes the transitive and reflexive closure of the relation $\Rightarrow$.

A Watson-Crick automaton accepts the language $L(M)$:

$$L(M) = \Big\{w_1 \in V^* | q_0 \wkdomain{w_1}{w_2} \Rightarrow^* f \genfrac{(}{)}{0pt}{1}{\lambda}{\lambda} \textnormal{ where } f \in F, w_2 \in V^*, \wkdomain{w_1}{w_2} \in WK_{\rho}(V)\Big\}$$

This means that only the upper strand is accepted by this automaton to the language $L$. The lower strand has just an auxiliary purpose.

\section{Special versions of Watson-Crick automata}
Four special versions of Watson-Crick automata are often used. These are:
\begin{itemize}
  \item{Stateless WKA -- The WKA has only one state: $Q = F = {q_0}$}
  \item{All final WKA -- All the states are final: $Q = F$}
  \item{Simple WKA -- each rule reads only one head: $(q({w_1 \atop w_2}) \rightarrow q' \in P) \Rightarrow (w_1 = \lambda \vee w_2 = \lambda)$}
  \item{1-limited WKA -- similar to Simple WKA but also reads only one symbol at a time: $(q({w_1 \atop w_2}) \rightarrow q' \in P) \Rightarrow |w_1 w_2| = 1$}
\end{itemize}

It has been shown that three of these four special types of WKAs have the same power as the actual WKA, namely all final WKA, simple WKA and 1-limited WKA (stateless WKA is weaker). Therefore one possible approach to decide membership would be to limit the decision algorithm to one of these three types without any loss in expressing power.

There are three different variants of deterministic WKA proposed in \cite{DETERM_WKA}. These are:
\begin{itemize}
  \item{Weakly deterministic WKA -- WKA where in each reachable configuration, there is at most one possible continuation.}
  \item{Deterministic WKA -- for any two rules which lead from the same state, either their upper strands or their lower strands must not be prefix comparable, meaning one is not the prefix of the other. Formally: $(q({u \atop v}) \rightarrow q_1 \in P \wedge q({u' \atop v'}) \rightarrow q_2 \in P) \Rightarrow u \nsim_p u' \vee v \nsim_p v'$ where $\sim_p$ is the relation of prefix comparability}
  \item{Stronly deterministic WKA}
\end{itemize}

It is not specified how to actually achieve weak determinism. In fact, \cite{DETERM_WKA} shows that this property is undecidable. Informally, for a WKA to be weakly deterministic but not deterministic, there must be at least two rules which could both be used in certain configuration (otherwise it would be deterministic). But such a configuration must not be reachable (otherwise it would not be weakly deterministic). The configuration may be unreachable trivially -- by such rules using an unreachable state or a symbols that have no related symbols in the complementarity relation. But a configuration may be unrechable non-trivially, if it is possible to tell how many symbols will be read from each strand before reaching certain state.

Both weakly deterministic and deterministic WKA are in reality not deterministic (in an intuitive sense). Their determinism relies on the fact, that the configuration is known. But that is probably not a typical way how to work with WKA, since WKA decides the membership in a language for the upper strand only. That means that a compatible lower strand has to be found in the process of running the WKA. Theoretically, it is possible to approach this problem by first generating all possible lower strands for the given upper strand based solely on the complementarity relation and afterwards use all these pairs as inputs for the WKA. In such a case the weakly deterministic and deterministic automata would be truly deterministic, however this is clearly not feasible for non-trivial complementarity relations. Therefore, the strongly deterministic WKA is the only one witch is truly deterministic under all circumstances, because the identity relation required leaves no space these types of non-determinism.

\section{Watson-Crick grammars}
There are several WK grammars.


\section{Watson-Crick Pushdown automata}
The Watson-Crick Pushdown automata (WCPDA) have been introduced in \cite{WK_PUSHDOWN_AUT}. It is basically a two-head pushdown automaton with the complementarity relation added on top. Formally a WCPDA $P$ is a 10-tuple $P = (Q, \#, \$, V, \Gamma, \delta, q_0, Z_0, F, \rho)$ with most symbols having the same standard meaning as in conventional Pushdown automaton -- $Q$ is a finite set of states, $V$ is an input alphabet, $\Gamma$ is a stack alphabet, $q_0 \in Q$ is a starting state, $Z_0 \in \Gamma$ is a starting stack symbol and $F \subseteq Q$ is the set of final states. Symbols $\#, \$ \notin V$ are left and right input markers on the two strands. $\rho$ is the complementarity relation similar to standard WKA.

$\delta$ is a set of rules in the following form: $(q, ({w_1 \atop w_2}), x) \rightarrow (q', \gamma) \textnormal{ where } q, q' \in Q, w_1, w_2 \in V^* \cup \#V^* \cup V^*\$ \cup \#V^*\$, x \in \Gamma, \delta \in \Gamma^*$. That means the automaton can transition from state $q$ to $q'$ reading the input $w_1$ with the first head and $w_2$ with the second and go to state $q'$ while putting a string (i.e. 0-n symbols) of the stack symbols onto the stack. The two strands on the input are enclosed in the beginning symbol $\#$ and the closing symbol $\$$, therefore the symbol $\#$ may appear in the begginning of $w_1$ or $w_2$ and similarly the closing symbol $\$$ at the end.




\section{Watson-Crick Context-free systems}

\section{Role of the complementarity relation}

\section{Comparison of expressing power of various models}

\chapter{Existing ways of testing membership in Watson-Crick languages}

\section{Using deterministic automata}

\section{WK-CYK}
The WK-CYK algorithm was introduced in \cite{WK_CYK} and it is an enhancement of the CYK algorithm modified for WK languages.
\subsection{The CYK algorithm}
The CYK algorithm, introduced in \todo{Look up the original paper} is used to decide the membership in a language defined by a context-free grammar which must be in Chomsky Normal Form.

There is a string and a grammar on the input. The algorithm uses bottom-up parsing. In CNF, there are two kinds of rules: $A \rightarrow a$ and $A \rightarrow BC$ where $A, B, C$ are non-terminals and $a$ is a terminal. In the first stage, it analyses the first kind of rules -- each of the symbols from the input string has to be generated by a rule or several rules of this form. Thus it gets a set of candidate non-terminals for each symbol.

In the next stage it uses the second kind of rules -- every non-terminal (except the starting one) has to be generated by such a rule. The algorithm is looking for rules which can generate the candidate non-terminals which have been found in the previous stage. All possible combinations need to be considered, for instance the sequence of non-terminals $ABC$ may be generated by rules $X \rightarrow AB$ and $Y \rightarrow XC$ or by rules $X \rightarrow BC$ and $Y \rightarrow AX$. In this way, the algorithm needs to find all possible ways to generate strings of increasing length (all parse trees). Finally, it needs to find a non-terminal that can generate the whole word and it must be the starting non-terminal in the given grammar. If it succeeds, the word given on the input is in the language, otherwise it is not.
A more detailed and formal description can be found in \todo{Look up the original paper}.

The complexity of the CYK algorithm is $O(n^4)$.

\todo{possibly show an example and the triangular visualization}

\subsection{Analysis of the WK-CYK algorithm}


\section{Using WK Pushdown automata}

\chapter{New algorithms for testing membership in Watson-Crick languages}

\section{Running non-deterministic WKA with BFA}

\section{Generating strands using CF WK grammar}

\chapter{Analysis of complexity of existing and new algorithms}

\chapter{Testing membership problem on chosen WK languages}

\chapter{Conclusion}
